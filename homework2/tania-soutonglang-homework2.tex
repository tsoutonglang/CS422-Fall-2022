% Options for packages loaded elsewhere
\PassOptionsToPackage{unicode}{hyperref}
\PassOptionsToPackage{hyphens}{url}
%
\documentclass[
]{article}
\usepackage{amsmath,amssymb}
\usepackage{lmodern}
\usepackage{iftex}
\ifPDFTeX
  \usepackage[T1]{fontenc}
  \usepackage[utf8]{inputenc}
  \usepackage{textcomp} % provide euro and other symbols
\else % if luatex or xetex
  \usepackage{unicode-math}
  \defaultfontfeatures{Scale=MatchLowercase}
  \defaultfontfeatures[\rmfamily]{Ligatures=TeX,Scale=1}
\fi
% Use upquote if available, for straight quotes in verbatim environments
\IfFileExists{upquote.sty}{\usepackage{upquote}}{}
\IfFileExists{microtype.sty}{% use microtype if available
  \usepackage[]{microtype}
  \UseMicrotypeSet[protrusion]{basicmath} % disable protrusion for tt fonts
}{}
\makeatletter
\@ifundefined{KOMAClassName}{% if non-KOMA class
  \IfFileExists{parskip.sty}{%
    \usepackage{parskip}
  }{% else
    \setlength{\parindent}{0pt}
    \setlength{\parskip}{6pt plus 2pt minus 1pt}}
}{% if KOMA class
  \KOMAoptions{parskip=half}}
\makeatother
\usepackage{xcolor}
\usepackage[margin=1in]{geometry}
\usepackage{color}
\usepackage{fancyvrb}
\newcommand{\VerbBar}{|}
\newcommand{\VERB}{\Verb[commandchars=\\\{\}]}
\DefineVerbatimEnvironment{Highlighting}{Verbatim}{commandchars=\\\{\}}
% Add ',fontsize=\small' for more characters per line
\usepackage{framed}
\definecolor{shadecolor}{RGB}{248,248,248}
\newenvironment{Shaded}{\begin{snugshade}}{\end{snugshade}}
\newcommand{\AlertTok}[1]{\textcolor[rgb]{0.94,0.16,0.16}{#1}}
\newcommand{\AnnotationTok}[1]{\textcolor[rgb]{0.56,0.35,0.01}{\textbf{\textit{#1}}}}
\newcommand{\AttributeTok}[1]{\textcolor[rgb]{0.77,0.63,0.00}{#1}}
\newcommand{\BaseNTok}[1]{\textcolor[rgb]{0.00,0.00,0.81}{#1}}
\newcommand{\BuiltInTok}[1]{#1}
\newcommand{\CharTok}[1]{\textcolor[rgb]{0.31,0.60,0.02}{#1}}
\newcommand{\CommentTok}[1]{\textcolor[rgb]{0.56,0.35,0.01}{\textit{#1}}}
\newcommand{\CommentVarTok}[1]{\textcolor[rgb]{0.56,0.35,0.01}{\textbf{\textit{#1}}}}
\newcommand{\ConstantTok}[1]{\textcolor[rgb]{0.00,0.00,0.00}{#1}}
\newcommand{\ControlFlowTok}[1]{\textcolor[rgb]{0.13,0.29,0.53}{\textbf{#1}}}
\newcommand{\DataTypeTok}[1]{\textcolor[rgb]{0.13,0.29,0.53}{#1}}
\newcommand{\DecValTok}[1]{\textcolor[rgb]{0.00,0.00,0.81}{#1}}
\newcommand{\DocumentationTok}[1]{\textcolor[rgb]{0.56,0.35,0.01}{\textbf{\textit{#1}}}}
\newcommand{\ErrorTok}[1]{\textcolor[rgb]{0.64,0.00,0.00}{\textbf{#1}}}
\newcommand{\ExtensionTok}[1]{#1}
\newcommand{\FloatTok}[1]{\textcolor[rgb]{0.00,0.00,0.81}{#1}}
\newcommand{\FunctionTok}[1]{\textcolor[rgb]{0.00,0.00,0.00}{#1}}
\newcommand{\ImportTok}[1]{#1}
\newcommand{\InformationTok}[1]{\textcolor[rgb]{0.56,0.35,0.01}{\textbf{\textit{#1}}}}
\newcommand{\KeywordTok}[1]{\textcolor[rgb]{0.13,0.29,0.53}{\textbf{#1}}}
\newcommand{\NormalTok}[1]{#1}
\newcommand{\OperatorTok}[1]{\textcolor[rgb]{0.81,0.36,0.00}{\textbf{#1}}}
\newcommand{\OtherTok}[1]{\textcolor[rgb]{0.56,0.35,0.01}{#1}}
\newcommand{\PreprocessorTok}[1]{\textcolor[rgb]{0.56,0.35,0.01}{\textit{#1}}}
\newcommand{\RegionMarkerTok}[1]{#1}
\newcommand{\SpecialCharTok}[1]{\textcolor[rgb]{0.00,0.00,0.00}{#1}}
\newcommand{\SpecialStringTok}[1]{\textcolor[rgb]{0.31,0.60,0.02}{#1}}
\newcommand{\StringTok}[1]{\textcolor[rgb]{0.31,0.60,0.02}{#1}}
\newcommand{\VariableTok}[1]{\textcolor[rgb]{0.00,0.00,0.00}{#1}}
\newcommand{\VerbatimStringTok}[1]{\textcolor[rgb]{0.31,0.60,0.02}{#1}}
\newcommand{\WarningTok}[1]{\textcolor[rgb]{0.56,0.35,0.01}{\textbf{\textit{#1}}}}
\usepackage{graphicx}
\makeatletter
\def\maxwidth{\ifdim\Gin@nat@width>\linewidth\linewidth\else\Gin@nat@width\fi}
\def\maxheight{\ifdim\Gin@nat@height>\textheight\textheight\else\Gin@nat@height\fi}
\makeatother
% Scale images if necessary, so that they will not overflow the page
% margins by default, and it is still possible to overwrite the defaults
% using explicit options in \includegraphics[width, height, ...]{}
\setkeys{Gin}{width=\maxwidth,height=\maxheight,keepaspectratio}
% Set default figure placement to htbp
\makeatletter
\def\fps@figure{htbp}
\makeatother
\setlength{\emergencystretch}{3em} % prevent overfull lines
\providecommand{\tightlist}{%
  \setlength{\itemsep}{0pt}\setlength{\parskip}{0pt}}
\setcounter{secnumdepth}{-\maxdimen} % remove section numbering
\ifLuaTeX
  \usepackage{selnolig}  % disable illegal ligatures
\fi
\IfFileExists{bookmark.sty}{\usepackage{bookmark}}{\usepackage{hyperref}}
\IfFileExists{xurl.sty}{\usepackage{xurl}}{} % add URL line breaks if available
\urlstyle{same} % disable monospaced font for URLs
\hypersetup{
  pdftitle={CS 422 Homework 2},
  pdfauthor={Tania Soutonglang},
  hidelinks,
  pdfcreator={LaTeX via pandoc}}

\title{CS 422 Homework 2}
\author{Tania Soutonglang}
\date{}

\begin{document}
\maketitle

{
\setcounter{tocdepth}{2}
\tableofcontents
}
\hypertarget{section}{%
\subsubsection{2.1}\label{section}}

\begin{Shaded}
\begin{Highlighting}[]
\FunctionTok{library}\NormalTok{(ISLR)}
\FunctionTok{data}\NormalTok{(Auto)}

\FunctionTok{set.seed}\NormalTok{(}\DecValTok{1122}\NormalTok{)}
\NormalTok{index }\OtherTok{\textless{}{-}} \FunctionTok{sample}\NormalTok{(}\DecValTok{1}\SpecialCharTok{:}\FunctionTok{nrow}\NormalTok{(Auto), }\FloatTok{0.95}\SpecialCharTok{*}\FunctionTok{dim}\NormalTok{(Auto)[}\DecValTok{1}\NormalTok{])}
\NormalTok{train.df }\OtherTok{\textless{}{-}}\NormalTok{ Auto[index,]}
\NormalTok{test.df }\OtherTok{\textless{}{-}}\NormalTok{ Auto[}\SpecialCharTok{{-}}\NormalTok{index, ]}
\end{Highlighting}
\end{Shaded}

\hypertarget{a}{%
\subsubsection{2.1-a}\label{a}}

\begin{Shaded}
\begin{Highlighting}[]
\NormalTok{autoModel }\OtherTok{\textless{}{-}} \FunctionTok{lm}\NormalTok{(mpg }\SpecialCharTok{\textasciitilde{}}\NormalTok{ . }\SpecialCharTok{{-}}\NormalTok{name, }\AttributeTok{data =}\NormalTok{ test.df)}
\end{Highlighting}
\end{Shaded}

\hypertarget{a-i}{%
\subsubsection{2.1-a-i}\label{a-i}}

Using name is not a good predictor because not all values in the data
set would have a similar or the same pattern of letters to make a
comparison without creating biased data.

\hypertarget{a-ii}{%
\subsubsection{2.1-a-ii}\label{a-ii}}

\begin{Shaded}
\begin{Highlighting}[]
\FunctionTok{summary}\NormalTok{(autoModel)}
\end{Highlighting}
\end{Shaded}

\begin{verbatim}
## 
## Call:
## lm(formula = mpg ~ . - name, data = test.df)
## 
## Residuals:
##      Min       1Q   Median       3Q      Max 
## -2.75902 -1.37162 -0.03106  1.11112  3.11388 
## 
## Coefficients:
##                Estimate Std. Error t value Pr(>|t|)    
## (Intercept)  -58.101140  26.223119  -2.216 0.046803 *  
## cylinders      0.669882   1.471494   0.455 0.657064    
## displacement  -0.016521   0.042538  -0.388 0.704540    
## horsepower     0.079434   0.074422   1.067 0.306816    
## weight        -0.008182   0.002388  -3.426 0.005023 ** 
## acceleration   0.423656   0.370316   1.144 0.274916    
## year           1.212136   0.266323   4.551 0.000665 ***
## origin        -0.668450   1.070896  -0.624 0.544183    
## ---
## Signif. codes:  0 '***' 0.001 '**' 0.01 '*' 0.05 '.' 0.1 ' ' 1
## 
## Residual standard error: 2.469 on 12 degrees of freedom
## Multiple R-squared:  0.9415, Adjusted R-squared:  0.9074 
## F-statistic: 27.61 on 7 and 12 DF,  p-value: 1.751e-06
\end{verbatim}

\begin{Shaded}
\begin{Highlighting}[]
\NormalTok{rss }\OtherTok{\textless{}{-}} \FunctionTok{round}\NormalTok{(}\FunctionTok{sum}\NormalTok{(autoModel}\SpecialCharTok{$}\NormalTok{residuals}\SpecialCharTok{\^{}}\DecValTok{2}\NormalTok{), }\AttributeTok{digit =} \DecValTok{2}\NormalTok{)}

\CommentTok{\# rse \textless{}{-} round(sqrt(mean(autoModel$residuals\^{}2)), digit = 2)}

\NormalTok{rmse }\OtherTok{\textless{}{-}} \FunctionTok{round}\NormalTok{(}\FunctionTok{sqrt}\NormalTok{(}\FunctionTok{mean}\NormalTok{(autoModel}\SpecialCharTok{$}\NormalTok{residuals}\SpecialCharTok{\^{}}\DecValTok{2}\NormalTok{)), }\AttributeTok{digit =} \DecValTok{2}\NormalTok{)}
\NormalTok{rmse}
\end{Highlighting}
\end{Shaded}

\begin{verbatim}
## [1] 1.91
\end{verbatim}

\begin{Shaded}
\begin{Highlighting}[]
\NormalTok{str }\OtherTok{\textless{}{-}} \FunctionTok{cat}\NormalTok{(}\FunctionTok{paste0}\NormalTok{(}\StringTok{"R{-}sq value is 0.93}\SpecialCharTok{\textbackslash{}n}\StringTok{Adjusted R{-}sq value is 0.92}\SpecialCharTok{\textbackslash{}n}\StringTok{RSR is }\SpecialCharTok{\textbackslash{}n}\StringTok{RMSE is "}\NormalTok{, rmse))}
\end{Highlighting}
\end{Shaded}

\begin{verbatim}
## R-sq value is 0.93
## Adjusted R-sq value is 0.92
## RSR is 
## RMSE is 1.91
\end{verbatim}

\hypertarget{a-iii}{%
\subsubsection{2.1-a-iii}\label{a-iii}}

\begin{Shaded}
\begin{Highlighting}[]
\NormalTok{autoRes }\OtherTok{\textless{}{-}}\NormalTok{ autoModel}\SpecialCharTok{$}\NormalTok{residuals}

\FunctionTok{plot}\NormalTok{(}\FunctionTok{fitted}\NormalTok{(autoModel), autoRes, }\AttributeTok{main =} \StringTok{"2.1{-}a{-}iii Auto Model Residuals"}\NormalTok{, }\AttributeTok{xlab =} \StringTok{"Model"}\NormalTok{, }\AttributeTok{ylab =} \StringTok{"Residuals"}\NormalTok{)}
\end{Highlighting}
\end{Shaded}

\includegraphics{tania-soutonglang-homework2_files/figure-latex/unnamed-chunk-4-1.pdf}

\hypertarget{a-iv}{%
\subsubsection{2.1-a-iv}\label{a-iv}}

\begin{Shaded}
\begin{Highlighting}[]
\FunctionTok{hist}\NormalTok{(autoRes, }\AttributeTok{main =} \StringTok{"2.1{-}a{-}iv Auto Module Residuals"}\NormalTok{, }\AttributeTok{xlab =} \StringTok{"Residuals"}\NormalTok{)}
\end{Highlighting}
\end{Shaded}

\includegraphics{tania-soutonglang-homework2_files/figure-latex/unnamed-chunk-5-1.pdf}

This histogram does not follow a Gaussian distribution. The distribution
of the residuals are primarily either below the regression line or
near/on the regression line.

\hypertarget{b-i}{%
\subsubsection{2.1-b-i}\label{b-i}}

According to the regression model made in part (a), weight, year, and
acceleration are statistically significant.

\begin{Shaded}
\begin{Highlighting}[]
\NormalTok{autoNewModel }\OtherTok{\textless{}{-}} \FunctionTok{lm}\NormalTok{(mpg }\SpecialCharTok{\textasciitilde{}}\NormalTok{ weight }\SpecialCharTok{+}\NormalTok{ year }\SpecialCharTok{+}\NormalTok{ acceleration, }\AttributeTok{data =}\NormalTok{ test.df)}
\end{Highlighting}
\end{Shaded}

\hypertarget{b-ii}{%
\subsubsection{2.1-b-ii}\label{b-ii}}

\begin{Shaded}
\begin{Highlighting}[]
\FunctionTok{summary}\NormalTok{(autoNewModel)}
\end{Highlighting}
\end{Shaded}

\begin{verbatim}
## 
## Call:
## lm(formula = mpg ~ weight + year + acceleration, data = test.df)
## 
## Residuals:
##     Min      1Q  Median      3Q     Max 
## -3.0478 -1.6948 -0.2249  1.7109  3.6604 
## 
## Coefficients:
##                Estimate Std. Error t value Pr(>|t|)    
## (Intercept)  -3.918e+01  1.491e+01  -2.628   0.0183 *  
## weight       -5.948e-03  6.431e-04  -9.248 8.05e-08 ***
## year          1.050e+00  1.760e-01   5.965 1.98e-05 ***
## acceleration  6.615e-02  2.266e-01   0.292   0.7741    
## ---
## Signif. codes:  0 '***' 0.001 '**' 0.01 '*' 0.05 '.' 0.1 ' ' 1
## 
## Residual standard error: 2.297 on 16 degrees of freedom
## Multiple R-squared:  0.9326, Adjusted R-squared:  0.9199 
## F-statistic: 73.76 on 3 and 16 DF,  p-value: 1.384e-09
\end{verbatim}

\begin{Shaded}
\begin{Highlighting}[]
\NormalTok{rmse }\OtherTok{\textless{}{-}} \FunctionTok{signif}\NormalTok{(}\FunctionTok{sqrt}\NormalTok{(}\FunctionTok{mean}\NormalTok{(autoNewModel}\SpecialCharTok{$}\NormalTok{residuals}\SpecialCharTok{\^{}}\DecValTok{2}\NormalTok{)), }\AttributeTok{digits =} \DecValTok{2}\NormalTok{)}

\NormalTok{str }\OtherTok{\textless{}{-}} \FunctionTok{cat}\NormalTok{(}\FunctionTok{paste0}\NormalTok{(}\StringTok{"R{-}sq value is 0.93}\SpecialCharTok{\textbackslash{}n}\StringTok{RSE is 2.3}\SpecialCharTok{\textbackslash{}n}\StringTok{RMSE is "}\NormalTok{, rmse))}
\end{Highlighting}
\end{Shaded}

\begin{verbatim}
## R-sq value is 0.93
## RSE is 2.3
## RMSE is 2.1
\end{verbatim}

The new model's values of R2, RSE, and RMSE did not change too much from
the original model.

\hypertarget{b-iii}{%
\subsubsection{2.1-b-iii}\label{b-iii}}

\begin{Shaded}
\begin{Highlighting}[]
\NormalTok{autoNewRes }\OtherTok{\textless{}{-}}\NormalTok{ autoNewModel}\SpecialCharTok{$}\NormalTok{residuals}

\FunctionTok{plot}\NormalTok{(}\FunctionTok{fitted}\NormalTok{(autoNewModel), autoNewRes, }\AttributeTok{main =} \StringTok{"2.1{-}b{-}iii Auto Model Residuals"}\NormalTok{, }\AttributeTok{xlab =} \StringTok{"Model"}\NormalTok{, }\AttributeTok{ylab =} \StringTok{"Residuals"}\NormalTok{)}
\end{Highlighting}
\end{Shaded}

\includegraphics{tania-soutonglang-homework2_files/figure-latex/unnamed-chunk-8-1.pdf}

\hypertarget{b-iv}{%
\subsubsection{2.1-b-iv}\label{b-iv}}

\begin{Shaded}
\begin{Highlighting}[]
\FunctionTok{hist}\NormalTok{(autoNewRes, }\AttributeTok{main =} \StringTok{"2.1{-}b{-}iv Auto Module Residuals"}\NormalTok{, }\AttributeTok{xlab =} \StringTok{"Residuals"}\NormalTok{)}
\end{Highlighting}
\end{Shaded}

\includegraphics{tania-soutonglang-homework2_files/figure-latex/unnamed-chunk-9-1.pdf}

This histogram does not follow a Gaussian distribution. The distribution
of the residuals are vary a lot throughout the model.

\hypertarget{b-v}{%
\subsubsection{2.1-b-v}\label{b-v}}

The new model would be better since the data is more centralized towards
the attributes with significantly smaller p-values, indicating a smaller
margin of error in the data.

\hypertarget{c}{%
\subsubsection{2.1-c}\label{c}}

\begin{Shaded}
\begin{Highlighting}[]
\NormalTok{predictions }\OtherTok{\textless{}{-}} \FunctionTok{c}\NormalTok{(}\FunctionTok{predict}\NormalTok{(autoNewModel))}

\NormalTok{df }\OtherTok{\textless{}{-}} \FunctionTok{data.frame}\NormalTok{()}
\end{Highlighting}
\end{Shaded}

\hypertarget{d}{%
\subsubsection{2.1-d}\label{d}}

\hypertarget{e}{%
\subsubsection{2.1-e}\label{e}}

\hypertarget{f}{%
\subsubsection{2.1-f}\label{f}}

\end{document}
